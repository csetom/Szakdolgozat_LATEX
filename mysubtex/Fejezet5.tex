\chapter{Adatok gyüjtése a mesterséges inteligencia betanításához}
\thispagestyle{fancy}
\pagestyle{fancy}


\section{Tervezés}
Ahhoz hogy a mesterséges inteligenciát betanítsuk, szükségünk lesz a lehető legtöbb információra, amit képesek vagyunk egy adott menetről összegyüjteni.
Egy adott játék a következőképpen néz ki:

\begin{itemize}
\item A játékos az eddigi tudása alapján, kiválasztja melyik sor és melyik oszlop kártyáját választja.
\item Felfordítja a kiválasztott kártyát, ekkor megtudja, milyen betű szerepel rajta. 
\item Ezen információ és az eddigi tudása alapján kiválasztja a következő kártyát a tábláról. Itt két lehetőssége van a játékosnak: 
\begin{itemize}
    \item Tudja vagy legalábbis nagy százalék valószínűséggel sejti, hogy melyik a párja a kártyának 
    \item Egyáltalán nem, vagy kis valószínűséggel tudja, hol a párja, emiatt felfordít direkt olyan kártyát, melyről eddig semmi ismerete nincs. Vagyis ahogy én hívom, felfedezi a pályát.
\end{itemize}
\item Ha párok voltak, eltűnnek a tábláról a kártyák.
\item A játékos ezt a folyamatot addig ismétli, míg el nem tűnik az összes kártya.
\end{itemize}

Amire szükségünk van a játék ismeretéhez tehát a következők: 
\begin{itemize}
    \item Hány kártya van a táblán. Mivel mindig egy $N \cdot N$ négynezetet rakunk ki, így ezt fixen ismerjük. 
    \item Milyen betűk szerepelnek a leosztásban
    \item Sorrendben, melyik sor és oszlop kártyái lettek felfodítva, és mi szerepelt a memória kártyán. 
\end{itemize}

    \centering
A következő adatszerkezetet fogom használni: 
\begin{figure}[h]

 \begin{lstlisting}
\{
  "data": { 
    "played_games": {
      "6": [
        {
          "card_labels": ['A','B','C','D','E','F'], 
          "card_pair_number": 6,
          "card_selections": [ 
            {
              "chosen_first": true,
              "label": "A",
              "x": 0,
              "y": 0
            },
            {
                ...
            },
          ]
        },
        {
            ...
        }
      ]
    }
  }
}
\end{lstlisting}
\end{figure}
\section{Próbálkozások}
\subsection{Kiküldeni exportálva}
\subsection{Oracle Cloud}
\subsection{Fly.io}
\section{Végleges megoldás}
