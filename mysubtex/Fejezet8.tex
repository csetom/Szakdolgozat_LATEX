\chapter{Összefoglaló}

\thispagestyle{fancy}
\pagestyle{fancy}

Összességében sikerült létrehoznom egy memóriajátékot a Godot Engine segítségével, és sikeresen begyűjtöttem elegendő tesztadatot ahhoz, hogy a játék működjön a mesterséges intelligencia ellen. A projekt során értékes tapasztalatokat szereztem, amelyek segítettek a program fejlesztésében.

A játék azonban még fejlődési lehetőségeket rejt magában. Elsősorban a mesterséges intelligencia modelljét érdemes lenne tovább finomítani. Esetleg egy összetettebb tanítási mintával is megpróbálhatnám betanítani, hogy a rendszer jobban alkalmazkodjon a játék dinamikájához. Emellett a kimeneti mintát is érdemes lenne javítani: ahelyett, hogy csak egy kártyát fordítana fel, a rendszer egy valószínűségi táblát is adhatna vissza, amely megmutatja, hogy a különböző kártyák milyen valószínűséggel kerülnek felfordításra a játék során. Ez a megközelítés növelheti az MI stratégiájának komplexitását és versenyképességét.

Továbbá az élvezeti faktort is érdemes lenne növelni. Például bevezethetünk egy pontrendszert, amely világosan jelzi, hogy a játékos és a mesterséges intelligencia hány ponttal zárta a játékot. Ez nemcsak az izgalmat fokozná, hanem motiválhatná a játékosokat a teljesítményük javítására is.

A projektből az a tanulság vonható le, hogy gyakran a legegyszerűbbnek tűnő feladatok jelentették a legnagyobb kihívást a projekt során. Az apró részletek és finomítások, amelyek elsőre jelentéktelennek tűntek, kulcsszerepet játszottak a játék élvezhetőségében és funkcionalitásában. Ezt a tapasztalatot a jövőbeli projekteknél is hasznosítani fogom.


