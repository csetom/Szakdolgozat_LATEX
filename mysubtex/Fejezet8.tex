\chapter{Összefoglaló}

\thispagestyle{fancy}
\pagestyle{fancy}
Összességében sikerült létrehoznom egy teljes értékű memóriajátékot a Godot Engine segítségével, amelyben a játékos a mesterséges intelligencia ellen mérheti össze tudását. A fejlesztési folyamat során sikeresen begyűjtöttem elegendő tesztadatot ahhoz, hogy a játék zökkenőmentesen működjön és valódi kihívást jelentsen a felhasználók számára. A projekt során értékes tapasztalatokat szereztem nemcsak a programozás és a játékfejlesztés terén, hanem a mesterséges intelligencia integrálásában és optimalizálásában is, amelyek nagymértékben segítettek a program továbbfejlesztésében és finomításában.

A játék azonban még számos fejlesztési lehetőséget rejt magában, amelyekkel tovább növelhető a játékélmény és a mesterséges intelligencia hatékonysága. Elsősorban a mesterséges intelligencia modelljét érdemes lenne tovább finomítani, hogy még intelligensebb és adaptívabb legyen a játékos lépéseire. Egy összetettebb és kiterjedtebb tanítási mintával is célszerű lenne betanítani a rendszert, hogy jobban alkalmazkodjon a játék dinamikájához és változatosabb stratégiákat tudjon alkalmazni. Emellett a kimeneti mintát is fejleszteni lehetne: ahelyett, hogy a rendszer csak egy kártyát fordítana fel, egy valószínűségi táblát is adhatna vissza. Ez a tábla megmutatná, hogy a különböző kártyák milyen valószínűséggel kerülnek felfordításra a játék során, ami jelentősen növelhetné az MI stratégiájának komplexitását és versenyképességét, ezzel még nagyobb kihívást jelentve a játékosok számára.

Továbbá az élvezeti faktort is érdemes lenne növelni különböző játékelemek bevezetésével. Például bevezethetünk egy részletes pontrendszert, amely nemcsak a győzelmet vagy vereséget jelzi, hanem pontosan mutatja, hogy a játékos és a mesterséges intelligencia hány ponttal zárta a játékot, és hogyan teljesítettek az egyes fordulókban. Ezen kívül különféle nehézségi szinteket is implementálhatnánk, amelyek lehetővé teszik a játékosok számára, hogy a saját képességeiknek megfelelő kihívást válasszanak. Ez nemcsak az izgalmat fokozná, hanem motiválná a játékosokat a teljesítményük folyamatos javítására és a játék mélyebb megismerésére is.

A projektből számos fontos tanulság vonható le. Elsősorban az, hogy gyakran a legegyszerűbbnek tűnő feladatok jelentették a legnagyobb kihívást a fejlesztés során. Az apró részletek és finomítások, amelyek elsőre jelentéktelennek tűntek, kulcsszerepet játszottak a játék élvezhetőségében, funkcionalitásában és a felhasználói élményben. Ezek a tapasztalatok rávilágítottak arra, hogy mennyire fontos a részletekre való odafigyelés és a folyamatos tesztelés a fejlesztési folyamatban. Ezt a megszerzett tudást és tapasztalatot a jövőbeli projektek során is hasznosítani fogom, hogy még magasabb színvonalú és élvezetesebb alkalmazásokat hozzak létre.