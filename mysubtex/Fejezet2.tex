\chapter{Irodalomkutatás}

\thispagestyle{fancy}
\pagestyle{fancy}

Kutatásom során első sorban azt akartam megvizsgálni, hogy mások milyen MI tanítási formákat használtak különböző játékok esetén. 

Sokáig úgy tartották, hogy az egyetlen játék, amit nem tudunk megtanítani a mesterséges inteligenciának, az a Go. Azonban, Training Deep Convolutional Neural Networks to Play Go tanulmánya \cite{pmlr-v37-clark15} bemutatja,
 hogy egy neurális hálózattal ez is lehetséges. A bemenethez az aktuális állapotot, az aktuális állást használják, a kimenethez pedig a valószínűségét az összes rácspontnak a táblán, ahova az MI lerakja a követ. 

 A neurális hálózatot használhatjuk különböző játékelméleti döntésekhez is, ahogy azt A recurrent neural network for game theoretic decision making \cite{bhatia2014recurrent} cikkben is olvashatjuk. 

A Modular Neural Networks for Learning Context-Dependent Game Strategies \cite{boyan1992modular} tanulmányban már 1991 -ben írtak arról, hogyan lehet megfigyelt tanítással kontextus függő játékokat játszatni a neurális hálózattal, például amőbát, vagy Backgammon-t. A Backgammonhoz moduláris és monolithic hálózatot is használnak. 

A Cognitive Learning and the Multimodal Memory Game: Toward Human-Level Machine Learning \cite{4634261} egy olyan módszer, amelyben egy memória játékot próbálnak egy MI-nak megtanítani olyan mintát használva, ahogyan az emberek is tanulnak.

A szakdolgozatomhoz végzett kutatás során azt vizsgáltam, hogyan alkalmazzák a mesterséges intelligencia tanítási módszereit különböző játékokban, és hogyan tudnám ezeket a technikákat a saját projektemhez adaptálni.
Az egyes tanulmányok bemutatása nemcsak arra világított rá, hogy milyen különféle megközelítések léteznek, hanem arra is, hogy ezek a módszerek miként alkalmazhatók a memóriajáték fejlesztése során. 
Ezáltal a szakdolgozatom elméleti hátterét erősítette, és gyakorlati szempontból is hozzájárult a saját MI modellem kidolgozásához és finomításához.