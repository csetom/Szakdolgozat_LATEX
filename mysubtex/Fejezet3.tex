\chapter{Felhasznált technológiák}

\thispagestyle{fancy}
\pagestyle{fancy}

Munkám során törekedtem arra, hogy a felhasznált technológiákat lehetőleg minimalizáljam. 
Figyelembe vettem továbbá azt is, hogy nyílt forráskódú, és multiplatform eszközöket válasszak. Ezen döntések lehetővé tették számomra a kellő flexibilitást, és elősegítették a munkámat.
\section{Godot Engine}
A Godot egy nyílt forráskódú, ingyenesen elérhető játékmotor és fejlesztői környezet, amelyet a játékok, interaktív tartalmak és egyéb multimédiás alkalmazások létrehozására terveztek. A motorot Juan Linietsky, Ariel Manzur és George Marques alapította 2014-ben, és azóta folyamatos fejlesztés alatt áll, számos kiadott verzióval és fejlesztői közösséggel.

A Godot kiemelkedik sokoldalúsága és könnyűsége miatt. Az egyik legfontosabb jellemzője az integrált fejlesztői környezet (IDE), amely segítségével a fejlesztők egyetlen alkalmazásban végezhetik el a játékterv készítését, a kódolást, a grafika létrehozását és a játéktesztek futtatását. Az IDE rendelkezik számos funkcióval, mint például kódszerkesztő, jelenet szerkesztő, animációkészítő, fizikai motor, hangkezelő, és még sok más, amelyek egyszerűsítik és gyorsítják a fejlesztési folyamatot.

A Godot támogatja a kettő dimenziós (2D) és a három dimenziós (3D) játékfejlesztést is, és számos előre elkészített funkciót és sablont kínál mindkét típushoz. A motor különösen erős a vizuális effektek, az animációk és a szkriptelés terén, és lehetővé teszi a fejlesztők számára, hogy rugalmasan alkalmazzák saját ötleteiket és terveiket a játék készítése során.
 
A Godotot széles körben használják különböző projektekben, beleértve az indie játékokat, oktatási alkalmazásokat, interaktív médiaalkotásokat és még sok mást. A motor aktív és elkötelezett fejlesztői közösséggel rendelkezik, amely folyamatosan hozzájárul az új funkciók, javítások és dokumentációk fejlesztéséhez.

Azért a Godot mellett döntöttem, mivel úgynevezett Assesst Libary formályában, lehetőségem volt könnyedén integrálni a projektembe a VR eszközök natív támogatását. Valamint hobbimból kifolyólag van ismeretem a program használatában.

% \section{OpenXR}
% Az OpenXR egy nyílt szabványú API (Application Programming Interface), amelyet a virtuális valóság és kiterjesztett valóság (AR) alkalmazások fejlesztésére terveztek. 

% Az OpenXR-t az OpenXR Working Group  hozta létre azért, amelyben olyan nagy szereplők vesznek részt, mint az Oculus, a Valve, az Epic Games és a Google.

% Az OpenXR célja, hogy egy általános API-t nyújtson, amely lehetővé teszi a fejlesztők számára, hogy alkalmazásaikat egységes kód alapján futtathassák az összes támogatott VR/AR eszközön, függetlenül azok gyártójától vagy típusától. 
% Ezáltal nincs szükség külön-külön optimalizálniuk alkalmazásokat minden egyes VR/AR platformra, hanem egyszerűen használhatják az OpenXR-t, hogy egyetlen kódbázisból több platformon is futtatható legyen a kész termékük.

% Az API lehetővé teszi a fejlesztők számára, hogy közvetlen hozzáférést kapjanak a VR/AR eszközök hardveres funkcióihoz és jellemzőihez, mint például a képminőség beállítás és a mozgásérzékelés.

% Az OpenXR API széles körben támogatott a VR/AR iparágban, és egyre több eszköz és platform támogatja az OpenXR specifikációkat, mint például a Godot.

% Ennek az API-nak hála, lényegében egy gombnyomásra kitudtam exportálni a Meta Quest 3 VR szemüvegemre a kész játékot, és futtatni tudtam rajta azt azonnal.
\section{CloudFlair}

\section{GDscript}
A GDScript a Godot engine saját szkriptelési nyelve, amelyet a játékfejlesztéshez terveztek. 
Könnyen tanulható és használható nyelv, amelyet kifejezetten a Godot-hoz optimalizáltak, így tökéletesen illeszkedik a motor által nyújtott funkciókhoz és struktúrához.

A GDScript egy dinamikus típusú script nyelv, ezáltal egyszerűbb és rugalmasabb kódolási stílust tesz lehetővé, amely könnyen alkalmazható a játékfejlesztés során.

Támogatja az objektumorientált programozás alapvető elveit, mint például az osztályok, az öröklődés és a polimorfizmus. Emellett rendelkezik számos beépített funkcióval és osztállyal, melyek jelentősen megkönnyítik a játékprogram elkészültét.
