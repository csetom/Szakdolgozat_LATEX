\chapter{Tesztelési eredmények}

\thispagestyle{fancy}
\pagestyle{fancy}

A témavezetőm segítségével a kész projektemet, mely már tartalmazta a mesterséges inteligenciát is,  sikerült az egyetem hallgatóival teszteltetni.
A teszt eredményeit a User Experience Questionnaire (Felhasználói Élmény Kérdőív, röviden UEQ) adat-elemző eszközzel, valamint a System Usability Scale (Rendszer Használhatósági Skála, azaz SUS) segítségével értékeltük ki. 

\section{Demográfiai adatok}
A szakdolgozatom tesztelő egyetemi hallgatók demográfiai adatok a \ref{tab:demografia} táblázatban láthatók. A tesztelés célja az alkalmazás funkcionalitásának és használhatóságának értékelése volt. A résztvevők visszajelzéseit gyűjtöttük, és ezen észrevételek alapján a jövőben tervezzük a program finomítását.

\vspace{1cm}

\begin{table}[!h]
    \caption{Demográfiai adatok}
    \label{tab:demografia}
    \resizebox{\linewidth}{!}{%
    \begin{tabular}{ccccccc} % specify the number of columns
    \hline
    \multicolumn{1}{c}{\textbf{ID}} & 
    \multicolumn{1}{c}{\textbf{Korcsoport}} & 
    \multicolumn{1}{c}{\textbf{Nem}} & 
    \multicolumn{1}{c}{\textbf{Legmagasabb iskola}} & 
    \multicolumn{1}{c}{\textbf{Techn. Járt.}} & 
    \multicolumn{1}{c}{\textbf{Elsődleges játékhoz}} & 
    \multicolumn{1}{c}{\textbf{Oktatásban töltött évek száma}} \\ \hline
    
    1 & 18-24 & Férfi & Középiskola & Haladó & Asztali számítógép & 13 \\
    2 & 25-34 & Férfi & Középiskola & Középhaladó & Laptop & 18 \\
    3 & 18-24 & Férfi & Technikus & Kezdő & Laptop & 14 \\
    4 & 18-24 & Férfi & Technikus & Haladó & Asztali számítógép & 13 \\
    5 & 18-24 & Férfi & Középiskola & Középhaladó & Asztali számítógép & 16 \\
    6 & 18-24 & Férfi & Középiskola & Kezdő & Asztali számítógép & 14 \\
    7 & 18-24 & Férfi & Középiskola & Középhaladó & Asztali számítógép & 15 \\
    8 & 18-24 & Férfi & Középiskola & Haladó & Asztali számítógép & 16 \\
    9 & 18-24 & Férfi & Középiskola & Haladó & Asztali számítógép & 13 \\
    10 & 25-34 & Férfi & Mesterfokozat & Szakértő & Okostelefon & 20 \\
    11 & 25-34 & Férfi & Főiskolai diploma & Szakértő & Asztali számítógép & 17 \\
    
    \hline
    \end{tabular}%
    }
    \end{table}

\section{UEQ - User Experience Questionnaire}


A \textit{User Experience Questionnaire} (UEQ) egy gyakran használt eszköz, amely lehetővé teszi a termékek és szolgáltatások felhasználói élményének mérését. Az UEQ kifejezetten úgy lett kialakítva, hogy gyorsan és hatékonyan nyújtson visszajelzést a felhasználói élmény különböző aspektusairól, mint például a használhatóság, az esztétika és az érzelmi hatások.

\subsection{Kérdőív felépítése és dimenziói}

Az UEQ kérdőív 26 tételből áll, amelyeket hat különböző dimenzióra osztanak:

\begin{itemize}
    \item \textbf{Vonzalom (Attractiveness)}: Az általános benyomás és az érzelmi reakció a termékre vagy szolgáltatásra.
    \item \textbf{Megértés (Perspicuity)}: A termék vagy szolgáltatás használatának egyszerűsége és könnyű megértése.
    \item \textbf{Hatékonyság (Efficiency)}: A feladatok gyors és hatékony végrehajtásának képessége.
    \item \textbf{Szervezettség (Dependability)}: Az irányítás érzete és a rendszer kiszámíthatósága.
    \item \textbf{Stimuláció (Stimulation)}: A termék vagy szolgáltatás által nyújtott inspiráció és motiváció.
    \item \textbf{Újdonság (Novelty)}: Az innováció érzete és a termék vagy szolgáltatás újszerűsége.
\end{itemize}

A következő megállapításokat tehetjük a különböző értékelésekről és azok szerepéről a mi esetünkben:

\begin{itemize}
    \item \textbf{Semleges értékelés (-0,8 és 0,8 között):} Ez a tartomány a skála viszonylag semleges megítélését jelzi. Ilyen értékek esetén az eredmények nem mutatnak kiemelkedő pozitív vagy negatív irányt.
    
    \item \textbf{Pozitív értékelés (> 0,8):} Ezek az értékek pozitív megítélést jelentenek. A mi esetünkben, ha a skálán +0,8 feletti értékek szerepelnek, az azt jelzi, hogy a vizsgált jellemzők kedvezőek és pozitív benyomást keltenek.
    
    \item \textbf{Negatív értékelés (< -0,8):} Ezek az értékek negatív értékelést tükröznek. Ha a skála -0,8 alatti értékeket mutat, az a vizsgált jellemzők kedvezőtlen megítélésére utal.
    
    \item \textbf{Skálatartomány:} A skála értékeinek teljes tartománya -3 (nagyon rossz) és +3 (nagyon jó) között mozog. A valós alkalmazásokban a legtöbb esetben a -2 és +2 közötti értékekkel találkozunk, mivel az extrém válaszok ritkán fordulnak elő.
\end{itemize}

\subsection{Kiértékelés}
\begin{table}[h]
    \centering
    \caption{UEQ skála, Átlag és szórás}
    \begin{tabular}{|l|c|c|}
        \hline
        \textbf{Skála} & \textbf{Átlag} & \textbf{Szórás} \\ \hline
        Kellem & 1.167 & 0.53 \\ \hline
        Áttekinthetőség & 2.364 & 0.74 \\ \hline
        Hatékonyság & 0.750 & 0.59 \\ \hline
        Megbízhatóság & 1.136 & 1.08 \\ \hline
        Ösztönzés & 1.159 & 0.63 \\ \hline
        Újszerűség & 0.500 & 0.78 \\ \hline
    \end{tabular}
    \label{tab:ueq_scales}
\end{table}

\begin{figure}[h]
    \center
    \includegraphics[width=0.75\textwidth]{img/UEQ_diagram.png}
    \caption{UEQ diagram eredmények}
    \label{diag:ueq}
\end{figure}

A \ref{tab:ueq_scales} táblázat adatai alapján, a hatból 4 dimenzióban jól teljesített  programom, Áttekinthetősségben kiemelkedően, és csak a Hatékonyság és az Újszerűség az, ahol neutrálisak az értékek. A \ref{diag:ueq}. ábra jól szemlélteti ezt. 


A UEQ skáláit a következő két fő kategóriába sorolhatjuk: pragmatikus minőség (amely magában foglalja az Áttekinthetőséget, Hatékonyságot és Megbízhatóságot) és hedonikus minőség (Ösztönzés és Újszerűség). 
Míg a pragmatikus minőség a feladatokhoz kapcsolódó minőségi szempontokat értékeli, addig a hedonikus minőség a nem feladatorientált minőségi tényezőkre fókuszál.

A \ref{tab:pragmatic_hedonic_quality}. táblázatban a \textit{kellemes} a három \textit{pragmatikus} és a két \textit{hedonikus} minőség szempont átlagértéke kerül kiszámításra. 

\begin{table}[h]
    \centering
    \caption{Pragmatikus és hedonikus minőségek}
    \begin{tabular}{|l|c|}
        \hline
        \textbf{Kategória} & \textbf{Érték} \\ \hline
        Kellem & 1.17 \\ \hline
        Pragmatikus minőség & 1.42 \\ \hline
        Hedonikus minőség & 0.83 \\ \hline
    \end{tabular}
    \label{tab:pragmatic_hedonic_quality}
\end{table}

Megvizsgálva az értékeket, boldogan tapasztaltam, hogy mind a három csoportban pozitív eredményt érhettem el, noha a Hedonikus értékekben épp csak megütöttem a 0.83-al. Ezt a \ref{diag:pragmatic_hedonic}. ábrában is láthattuk.

Kimagaslóan jó eredményem lett a pragmatikus minőségekből, amiből én arra következtettem, hogy a programom a feladatát megfelelően végzi el. 

A memóriajáték fejlesztése során a jövőben kiemelt figyelmet kell fordítani az újszerűségre, mivel ez hozzájárulhat a hedonikus minőség javításához is. Ezt valamilyen innovatív megoldással tudnám a legkönnyebben elérni.

\begin{figure}[h]
    \center
    \includegraphics[width=0.75\textwidth]{img/UEQ_pragmatic_hedonic.png}
    \caption{Diagram: UEQ pragmatikus és hedonikus minőségek}
    \label{diag:pragmatic_hedonic}
\end{figure}

\section{SUS - System Usability Scale}

A SUS (System Usability Scale), magyarul Rendszerhasználhatósági Skála, egy rövid, tíz kérdésből álló kérdőív, amely a rendszerek, termékek vagy szolgáltatások használhatóságának gyors és hatékony értékelésére szolgál. 
A kérdőív célja, hogy visszajelzést adjon a felhasználói élményről és az általános elégedettségről.

\subsection{Működési Elv}
\begin{enumerate}
    \item \textbf{Kérdések:} A SUS kérdőív 10 állítást tartalmaz, amelyek a rendszer használhatóságával kapcsolatosak. A kérdések között találhatóak pozitív (pl. ``Azt hiszem, szeretném gyakran használni ezt a rendszert.'') és negatív (pl. ``Azt tapasztaltam, hogy a rendszer szükségtelenül bonyolult.'') megfogalmazású kijelentések.
    
    \item \textbf{Értékelés:} A válaszadók 5 pontos skálán (1-től 5-ig) értékelik az állításokat. A skála 1 = ``Teljesen nem értek egyet'' és 5 = ``Teljesen egyetértek'' közötti értékeket használja.
    
    \item \textbf{Számítás:} A válaszok értékelése során a pozitív állításoknál 1-gyel csökkentik a adott értékelést, majd ezt az értéket kivonják 5-ből. A negatív állításoknál az értékelést egyszerűen levonják 1-ből. A végső pontszám kiszámításához az összes állítás eredményeit összeadják, majd ezt az értéket megszorozzák 2.5-tel, így egy 0-tól 100-ig terjedő skálán kaphatunk eredményt.
    
    \item \textbf{Értelmezés:} Az eredményként kapott pontszám a termék vagy rendszer általános használhatóságának értékelését szolgálja. A 68 körüli átlagpontszám jó használhatóságot jelez, míg az ennél alacsonyabb értékek esetében problémákra lehet következtetni.
\end{enumerate}

\subsection{Kérdések és a válaszok}
A tíz kérdésem/kijelentésem, a következők voltak:

\begin{enumerate}
    \item Azt hiszem, szeretném gyakran használni ezt a rendszert.
    \item Azt tapasztaltam, hogy a rendszer szükségtelenül bonyolult.
    \item Úgy gondoltam, hogy a rendszer könnyen használható.
    \item Azt hiszem, szükségem lenne egy műszaki szakember támogatására ahhoz, hogy használni tudjam ezt a rendszert.
    \item Úgy találtam, hogy a rendszer különböző funkciói jól integráltak.
    \item Úgy éreztem, hogy túl sok ellentmondás van ebben a rendszerben.
    \item Azt képzelem, hogy a legtöbb ember nagyon gyorsan megtanulná használni ezt a rendszert.
    \item Azt tapasztaltam, hogy a rendszer nagyon nehézkes a használat során.
    \item Nagyon magabiztosan éreztem magam a rendszer használata közben.
    \item Sokat kellett tanulnom, mielőtt elkezdhettem volna használni ezt a rendszert.
\end{enumerate}

\begin{table}[h]
    \centering
    \begin{tabular}{|c|c|c|c|c|c|c|c|c|c|c|}
        \hline
        ID & 1 & 2 & 3 & 4 & 5 & 6 & 7 & 8 & 9 & 10 \\ \hline
        1  & 2 & 4 & 2 & 2 & 3 & 2 & 4 & 2 & 4 & 2  \\ \hline
        2  & 4 & 1 & 5 & 1 & 4 & 1 & 5 & 1 & 5 & 1  \\ \hline
        3  & 3 & 1 & 5 & 1 & 3 & 1 & 5 & 1 & 5 & 1  \\ \hline
        4  & 3 & 1 & 5 & 1 & 4 & 1 & 5 & 1 & 4 & 1  \\ \hline
        5  & 5 & 1 & 1 & 2 & 2 & 3 & 2 & 2 & 3 & 2  \\ \hline
        6  & 2 & 2 & 4 & 2 & 3 & 2 & 5 & 1 & 4 & 1  \\ \hline
        7  & 5 & 1 & 5 & 1 & 5 & 1 & 5 & 1 & 5 & 1  \\ \hline
        8  & 4 & 1 & 5 & 3 & 3 & 1 & 5 & 1 & 5 & 1  \\ \hline
        9  & 1 & 1 & 5 & 1 & 5 & 1 & 5 & 1 & 5 & 1  \\ \hline
        10 & 1 & 1 & 5 & 2 & 5 & 1 & 5 & 1 & 5 & 1  \\ \hline
        11 & 2 & 1 & 5 & 3 & 3 & 2 & 4 & 1 & 5 & 1  \\ \hline
    \end{tabular}
    \caption{Válaszok}
    \label{tab:sus_data}
\end{table}

A kiértékeléshez a System Usability Scale Analysis Toolkit-et \cite{SUSAnaly62:online} használtam, mellyel a következő eredmények jöttek ki a \ref{tab:sus_data}. táblázat értékeire. 

\begin{itemize}
    \item \textbf{SUS Tanulmány Pontszám:} 82.5
    \item \textbf{Medián:} 87.5
    \item \textbf{Szórás:} 14.19
    \item \textbf{Jellemző:} Kiváló
    \item \textbf{Osztályzat:} A
    \item \textbf{Elfogadhatóság:} Elfogadható
    \item \textbf{Negyed:} 4. negyed
\end{itemize}

\begin{figure}[h]
    \center
    \includegraphics[width=\textwidth]{img/single_study_plot.png}
    \caption{SUS táblázat kiértékelt eredményei}
    \label{diag:sus_result}
\end{figure}

A diagrammon (\ref{diag:sus_result}. ábra) jól látszik, hogy az első és az ötödik kérdésekben a memória játékom alul teljesített. Az első kérdés, amely arra vonatkozik, hogy a felhasználók mennyire szeretnék gyakran használni a játékot, azt mutatja, hogy a válaszadók viszonylag alacsony érdeklődést mutatnak a rendszer rendszeres használata iránt.
 Ez aggasztó, mivel a gyakori használat egy fontos indikátora lehet a játék élvezhetőségének és vonzerejének.

Az ötödik kérdés kapcsán a felhasználók átlagos minőségűnek ítélték a megvalósítást, ami arra utal, hogy a játék nem hagyott maradandó benyomást, és nem tudta felkelteni a felhasználók érdeklődését olyan mértékben, hogy kiemelkedő minőséget mutasson.
 Ez a fajta közepes értékelés azt jelzi, hogy bár a játék elér egy bizonyos szintet, van még lehetőség a fejlődésre.

Mindkét területen – a felhasználói élmény fokozásában és a játék minőségének javításában – jelentős potenciál rejlik.
Fontos lenne részletesen elemezni, hogy mik azok a konkrét elemek, amelyek nem nyerték el a felhasználók tetszését.
Lehetséges, hogy a játékmenet, a grafika vagy akár a hangzásvilág is hozzájárulhat ehhez az érzéshez.
A felhasználói visszajelzések figyelembevételével, valamint alapos tesztelésekkel képesek lehetünk a szükséges módosításokat végrehajtani, így a játék vonzereje és használhatósága jelentősen javulhat.

Ezeket a problémákat kezelve, a jövőbeni fejlesztések során érdemes lenne a felhasználói élményt a középpontba állítani, hiszen egy élvezetes és vonzó játék ösztönzi a felhasználókat a rendszeres visszatérésre és a játék gyakori használatára
