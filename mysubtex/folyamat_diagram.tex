    \begin{tikzpicture}[>=latex',node distance=2cm,auto]
        % Define block styles
        \tikzstyle{block} = [rectangle, draw, text width=3cm, text centered, minimum height=1.5cm ]
        \tikzstyle{line} = [draw, -latex']

        % Nodes
        \node [block] (kutatomunka) {Kutatómunka};
        \node [block, below of=kutatomunka, node distance=3cm] (tervezes) {Játék megtervezése};
        \node [block, below of=tervezes, node distance=3cm] (jatek_megirasa) {Játék lefejlesztése};
        \node [block, below of=jatek_megirasa, node distance=3cm] (adatgyujtes) {Adatgyűjtés};
        \node [block, below of=adatgyujtes, node distance=3cm] (AI) {AI betanítása};
        \node [block, below of=AI, node distance=3cm] (AI_game) {AI játszatása};
        \node [block, right of=AI_game, node distance=5cm] (AI_ember) {AI játszatása ember ellen};
        \node [block, right of=adatgyujtes, node distance=5cm] (VR) {VR támogatás};

        % \node [block, right of=section1, node distance=4cm] (subsection1) {Subsection 1};
        % \node [block, right of=section2, node distance=4cm] (subsection2) {Subsection 2};

        % Arrows
        \path [line] (kutatomunka) -- (tervezes);
        \path [line] (tervezes) -- (jatek_megirasa);
        \path [line] (jatek_megirasa) -- (VR);
        \path [line] (jatek_megirasa) -- (adatgyujtes);
        \path [line] (adatgyujtes) -- (AI);
        \path [line] (AI) -- (AI_game);
        \path [line] (AI) -- (AI_ember);

    \end{tikzpicture}