\chapter{Mellékletek}
\thispagestyle{fancy}
\pagestyle{fancy}

\thispagestyle{fancy}
\pagestyle{fancy}

\subsubsection{Beadott fájlok}
\noindent 3D-memoria-jatek-mestereseges-inteligenciaval mappa (a szoftver fájljai):\\
\begin{tabular}{l l}
    \quad icon.svg  & \quad Godot alapértelmezett ikonja a játékhoz \\ 
    \quad icon.svg.import  & \quad Godot metaadat \\ 
    \quad openxr\_action\_map.tres & \quad OpenXR metaadat \\ 
    \quad project.godot  & \quad Godot projekt fájl \\ 
\end{tabular}
\\\\
\noindent 3D-memoria-jatek-mestereseges-inteligenciaval/Animations mappa (a szoftver animációs fájlok):\\
\begin{tabular}{l l}
    \quad Idle.res  & \quad Kártya animáció \\ 
    \quad Megfordit.res  & \quad Kártya megfordít animáció \\ 
    \quad RESET.res  & \quad Kártya Reset animáció \\
\end{tabular}
\\\\
\noindent 3D-memoria-jatek-mestereseges-inteligenciaval/Scenes mappa (a szoftver jelenetei):\\
\begin{tabular}{l l} 
    \quad MainScrenes  & \quad Fő jelenet\\ 
    \quad MenuScene.tscn  & \quad Főmenü jelenet \\ 
    \quad XrOrigin.tscn  & \quad XR kezdőjelenet \\ 
    \quad button.tscn  & \quad Gomb jelenet \\ 
    \quad plainField.tscn  & \quad Üres tér jelenet \\ 
    \quad plain\_camera.tscn  & \quad Kamera jelenet \\ 
    \quad basics.tscn  & \quad Alapok jelenet \\ 
    \quad Card.gdshader  & \quad Kártya shader\\ 
    \quad Card.tscn  & \quad Kártya jelenet\\ 
    \quad Outline.gdshader  & \quad Kártya körvonal shader\\ 
\end{tabular}

\newpage

\noindent 3D-memoria-jatek-mestereseges-inteligenciaval/Scripts mappa (a szoftver GDScript szkriptjei):\\
\begin{tabular}{l l}
    \quad Card.gd  & \quad Kártya jelenet szkriptje \\ 
    \quad Deck.gd  & \quad Paklik jelenet szkriptje \\ 
    \quad InputField.gd  & \quad Bemeneti mező szkript \\ 
    \quad MenuScene.gd  & \quad Menü jelenet szkriptje \\ 
    \quad VrScript.gd  & \quad VR specifikus szkript \\ 
    \quad basics.gd  & \quad Alap jelent szrikptje \\ 
    \quad button.gd  & \quad Gomb jelent szriptje \\ 
    \quad constant.gd  & \quad Állandók definiálásra használt szkript \\ 
    \quad deck\_timer.gd  & \quad Pakli időzítő szkript \\ 
    \quad http\_client.gd  & \quad HTTP kérések kezelése szkript \\ 
    \quad player\_data.gd  & \quad Játékos adatainak tárolására használt adatszerkezet definiáló szript \\ 
    \quad scene\_manager.gd  & \quad Jelenet váltást kezelő szkript \\ 
\end{tabular}
\\\\
\noindent 3D-memoria-jatek-mestereseges-inteligenciaval/Shaders mappa (A játék grafikai megjelenítését szolgáló fájlok): \\
\begin{tabular}{l l}
    \quad MenuScene.tres  & \quad Menű kinézetét szolgáló shader könyvtár\\ 
    \quad cardTestLibary.tres  & \quad Egy teszt saját shader könyvtár tesztje \\ 
    \quad new\_standard\_material\_3d.tres  & \quad A Gombokhoz használt shader könyvtár \\     
\end{tabular}
\\\\
\noindent 3D-memoria-jatek-mestereseges-inteligenciaval/Templates mappa: \\
\begin{tabular}{l l}
    \quad custom\_build\_template.html  & \quad A HTML exporthoz használt saját html template\\ 
\end{tabular}
\\
\\
\noindent 3D-memoria-jatek-mestereseges-inteligenciaval/fileServer mappa: (Adatok gyüjtését szolgáló fájlszerver) \\
\begin{tabular}{l l}
    \quad Dockerfile  & \quad Docker konténer buildelését szolgáló fájl \\ 
    \quad docker-compose.yaml  & \quad Docker kontérek kezelését szolgáló compose fájl \\ 
    \quad package-lock.json  & \quad Node.JS package-lock fájl, npm használja\\ 
    \quad package.json  & \quad Node.JS package fájl, npm használja. \\ 
    \quad server.js  & \quad Fájlszerver forráskódja \\ 
\end{tabular}

\newpage

\noindent 3D-memoria-jatek-mestereseges-inteligenciaval/tensorflow: (Az MI modell használatához és tanításához szükséges fájlok.) \\
\begin{tabular}{l l}
    \quad json\_to\_sha256.py  & \quad JSON adatok SHA256 konvertálását szolgáló Python szkript \\ 
    \quad my\_model.keras  & \quad A betanított és  mentett MI modell \\ 
    \quad sha256\_to\_binary.py  & \quad SHA256 hash konvertálása bitekre Python szkript \\ 
    \quad tensor.py  & \quad TensorFlow tanító algoritmus, MI betanítása szkript \\ 
    \quad use\_model.py  & \quad Flask szerver, az MI modell használata Python szkript\\ 
\end{tabular}


\noindent p8mqg2\_tex mappa (tex fájlok és képek): \\
\begin{tabular}{l l}

\end{tabular}
\vspace{28pt}

