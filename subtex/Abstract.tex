\textbf{\large{Tartalmi összefoglaló}}\\[32pt]

\thispagestyle{fancy}
\pagestyle{fancy}



\vspace{8pt}

% Tartalmi összefoglaló magyarul. Az összefoglalónak tartalmaznia kell (rövid, velős és összefüggő megfogalmazásban) a következőket:
% \begin{itemize}
%     \item téma megnevezése,
%     \item megoldott feladat megfogalmazása,
%     \item megoldási mód,
%     \item elért eredmények,
%     \item kulcsszavak (4-6 darab).
% \end{itemize}
% A tartalmi összefoglaló terjedelme nem lehet több egy A4-es oldalnál.\par
% Az összefoglalót magyar és angol nyelven kell készíteni. Sorrendben a dolgozat nyelvével megegyező kerül előrébb. A cím Title stílusú, formázása: Times New Roman/ Computer Modern, nagybetű, 14 pt, félkövér, középre igazított; az összefoglaló szövege Normál stílusú, formázása: Times New Roman, 12 pt, sorkizárt, 1.5-ös sortávolság.

A témám egy 3D memóriajáték mesterséges intelligenciával.
A projektemben létrehoztam egy 3D memóriajátékot a Godot Engine nevű játékmotorban, a saját nyelvének, a GDScriptnek a felhasználásával. A játékba lefejlesztettem választható nehézségi szinteket, a kártyákon pedig véletlenszerű betűpárok találhatók.
A játékot exportáltam HTML5 formátumba, és az itch.io nevű tartalommegosztó oldalra telepítettem.
Ezután létrehoztam egy fájlszervert. Erre a fájlszerverre elküldtem a játékosok adatait egy HTTP kérés segítségével minden játék után. Ehhez a Cloudflare-t, a Zero Tunnelt és a saját domainnevemet használtam, hogy fix néven érjem el a szerveremet.
A szerver Node.js-ben íródott, ExpressJS-t használva. A fájlszerver célja az volt, hogy a kapott adatokat lementsem a számítógépemre.
A fájlszerver és a Zero Tunnel Docker-konténerben futottak közös Docker-hálózaton.
A játék itch.io URL-jét elküldtem ismerősöknek, hogy minél több adatot gyűjtsek.
Az adatok tartalmazták a játékosok általuk megadott vagy generáltan kapott azonosítóját, valamint azt, hogy melyik nehézségi szinten, milyen leosztásban, melyik kártyákat választották.
A gyűjtött adatokkal betanítottam egy neurális hálót a TensorFlow Python-alapú szoftvercsomag felhasználásával.
Ehhez egy 4 rétegű neurális hálót használtam, ahol a bemeneti réteg a játék során előforduló összes kártya leképezése volt egy SHA256 hashfüggvénnyel.
A kimeneti réteg 2 neuronból állt, mely megadta  a következő felfordítandó kártya X és Y koordinátáját az asztalon, amelyet az AI a legjobbnak ítélt.
A két rejtett réteghez egy 128 és egy 64 node-ból álló dense réteget használtam ReLU aktivációs függvénnyel.
Ezután összekötöttem a játékot a betanított MI-modellel egy Flask Python-webszerver segítségével. A játékos húzása után, elküldtem minden eddigi válaszott kártyát a modellnek, és így megkaptam az MI lépését.
A játékot leteszteltem tizenegy emberrel. A felhasználói élményükről egy SUS és egy UEQ kérdőívet töltettem ki velük, melynek végeredményét kiértékeltem.
\vspace{8pt}



\textbf{Kulcsszavak: }3D memóriajáték, mesterséges intelligencia, Godot Engine, Cloudflare, neurális háló, TensorFlow
\newpage



\textbf{\large{Abstract}}\\[32pt]

\thispagestyle{fancy}
\pagestyle{fancy}

\vspace{8pt}
My topic is a 3D memory game with artificial intelligence.
In my project, I created a 3D memory game using the Godot Engine game engine, utilizing its own language, GDScript. I developed selectable difficulty levels in the game, and the cards featured random pairs of letters.
I exported the game in HTML5 format and deployed it on the content-sharing site called itch.io.
Next, I created a file server. The memory game sent automaticly the player data to this file server via an HTTP Request after each game. I used Cloudflare, Zero Tunnel, and my own domain name to access my server at a fixed address.
The server was written in Node.js, using ExpressJS. The purpose of the file server was to save the received data to my computer.
Both the file server and Zero Tunnel ran in Docker containers on a shared Docker network.
I sent the game’s itch.io URL to acquaintances to gather as much data as possible.
The data included the players’ provided or auto-generated IDs, as well as which difficulty level they played, the card layouts, and which cards they selected.
Using the collected data, I trained a neural network with the TensorFlow Python-based software package.
I used a 4-layer neural network for this, where the input layer was a mapping of all the cards that was selected during the game using an SHA256 hash function.
The output layer consisted of 2 neurons, which provided the X and Y coordinates of the next card to flip on the table that the AI deemed best.
For the two hidden layers, I used dense layers with 128 and 64 nodes, employing the ReLU activation function.
Then I connected the game with the trained AI model using a Flask Python web server.
After the player’s move, I sent all the previously selected cards to the model, and thus obtained the AI’s move.
I tested the game with eleven people. I had them fill out an SUS and a UEQ questionnaire about their user experience, which I ultimately evaluated.
\vspace{8pt}


\textbf{Keywords: }3D memory game, artificial intelligence, Godot Engine, Cloudflare, neural network, TensorFlow