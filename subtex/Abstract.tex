\textbf{\large{Tartalmi összefoglaló}}\\[32pt]

\thispagestyle{fancy}
\pagestyle{fancy}



\vspace{8pt}

% Tartalmi összefoglaló magyarul. Az összefoglalónak tartalmaznia kell (rövid, velős és összefüggő megfogalmazásban) a következőket:
% \begin{itemize}
%     \item téma megnevezése,
%     \item megoldott feladat megfogalmazása,
%     \item megoldási mód,
%     \item elért eredmények,
%     \item kulcsszavak (4-6 darab).
% \end{itemize}
% A tartalmi összefoglaló terjedelme nem lehet több egy A4-es oldalnál.\par
% Az összefoglalót magyar és angol nyelven kell készíteni. Sorrendben a dolgozat nyelvével megegyező kerül előrébb. A cím Title stílusú, formázása: Times New Roman/ Computer Modern, nagybetű, 14 pt, félkövér, középre igazított; az összefoglaló szövege Normál stílusú, formázása: Times New Roman, 12 pt, sorkizárt, 1.5-ös sortávolság.

A témám egy 3D memóriajáték mesterséges intelligenciával. A projektemben létrehoztam egy 3D-s memóriajátékot a Godot Engine nevű játékmotorban, a saját nyelvének, a GDScriptnek a felhasználásával. A játékot kiexportáltam HTML5 formátumba, és az itch.io nevű tartalommegosztó oldalra telepítettem.

Ezután létrehoztam egy fájlszervert. Erre a fájlszerverre elküldtem a játékosok adatait egy HTTP Request segítségével minden játék után. Ehhez a Cloudflare-t, a Zero Tunnelt és a saját domainnevemet használtam, hogy fix néven érjem el a szerveremet.

A szerver NodeJS-ben íródott, ExpressJS-t használva. A fájlszerver célja az volt, hogy a kapott adatokat lementsem. A fájlszerver és a Zero Tunnel Docker-konténerben futottak közös Docker-hálózaton.

A játék itch.io URL-jét elküldtem ismerősöknek, hogy minél több adatot gyűjtsek.

A gyűjtött adatokkal betanítottam egy neurális hálót a TensorFlow Python-alapú szoftvercsomag felhasználásával. Ehhez egy 4 rétegű neurális hálót használtam, ahol a bemeneti réteg a játék során előforduló összes kártya leképezése volt egy SHA256 hashfüggvénnyel. A kimeneti réteg a következő felfordítandó kártya volt, amelyet az AI a legjobbnak ítélt.

Ezután összekötöttem a játékot a betanított MI-modellel egy Flask Python-webszerver segítségével. A játékot leteszteltem tizenegy emberrel. A felhasználói élményükről egy SUS és egy UEQ kérdőívet kitöltettem velük, melynek végeredményben kiértékeltem.
\vspace{8pt}



\textbf{Kulcsszavak: }3D memóriajáték, mesterséges intelligencia, Godot Engine, GDScript, HTML5, itch.io, fájlszerver, HTTP Request, Cloudflare, Zero Tunnel, NodeJS, ExpressJS, Docker, neurális háló, TensorFlow, SHA256, Flask, SUS, UEQ, tesztelés, adatgyűjtés
\newpage



\textbf{\large{Abstract}}\\[32pt]

\thispagestyle{fancy}
\pagestyle{fancy}

\vspace{8pt}

My topic is a 3D memory game with artificial intelligence. In my project, I created a 3D memory game using the Godot Engine game engine, utilizing its own language, GDScript. I exported the game into HTML5 format and deployed it on a content-sharing site called itch.io.

After that, I created a file server. I sent the players’ data to this file server using an HTTP request after each game. To achieve a fixed name for accessing my server, I used Cloudflare, Zero Tunnel, and my own domain name.

The server was written in NodeJS using ExpressJS. The purpose of the file server was to save the received data. Both the file server and Zero Tunnel ran in Docker containers on a shared Docker network.

I sent the itch.io URL of the game to acquaintances to collect as much data as possible.

Using the collected data, I trained a neural network with the TensorFlow Python-based software package. For this, I used a 4-layer neural network where the input layer was a mapping of all cards occurring during the game using an SHA256 hash function. The output layer was the next card to be flipped, which the AI deemed the best.

Then I connected the game with the trained AI model using a Flask Python web server. I tested the game with eleven people and wrote an evaluation of their user experience. In the end, I analyzed this data.
\vspace{8pt}


\textbf{Keywords: }3D memory game, artificial intelligence, Godot Engine, GDScript, HTML5, itch.io, file server, HTTP Request, Cloudflare, Zero Tunnel, NodeJS, ExpressJS, Docker, neural network, TensorFlow, SHA256, Flask, SUS, UEQ, testing, data collection.