\textbf{\large{Tartalmi összefoglaló}}\\[32pt]

\thispagestyle{fancy}
\pagestyle{fancy}



\vspace{8pt}

Tartalmi összefoglaló magyarul. Az összefoglalónak tartalmaznia kell (rövid, velős és összefüggő megfogalmazásban) a következőket:
\begin{itemize}
    \item téma megnevezése,
    \item megoldott feladat megfogalmazása,
    \item megoldási mód,
    \item elért eredmények,
    \item kulcsszavak (4-6 darab).
\end{itemize}
A tartalmi összefoglaló terjedelme nem lehet több egy A4-es oldalnál.\par
Az összefoglalót magyar és angol nyelven kell készíteni. Sorrendben a dolgozat nyelvével megegyező kerül előrébb. A cím Title stílusú, formázása: Times New Roman/ Computer Modern, nagybetű, 14 pt, félkövér, középre igazított; az összefoglaló szövege Normál stílusú, formázása: Times New Roman, 12 pt, sorkizárt, 1.5-ös sortávolság.

\vspace{8pt}



\textbf{Kulcsszavak: } Kulcsszó1, kulcsszó2, kulcsszó3, kulcsszó4, kulcsszó5, kulcszó6

\newpage



\textbf{\large{Abstract}}\\[32pt]

\thispagestyle{fancy}
\pagestyle{fancy}

\vspace{8pt}

Tartalmi összefoglaló angol nyelven, a tartalma és formázása megegyezik a magyar nyelvű tartalmi összefoglalóval.

\vspace{8pt}


\textbf{Keywords: } Keyword1, Keyword2, Keyword3, Keyword4, Keyword5, Keyword6 