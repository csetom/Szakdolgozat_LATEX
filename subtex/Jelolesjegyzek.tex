
%\nomenclature{$U$}{Feszültség}
%\nomenclature{$\boldsymbol{x}$}{Állapotvektor}
%\nomenclature{$\boldsymbol{A}$}{Incidencia mátrix}


\nomenclature{$AI$}{Artificial Intelligence (Mesterséges Intelligencia)}
\nomenclature{$3D$}{Három dimenziós }
\nomenclature{$VR$}{Virtual Realty (VR) }

\cleardoublepage
\markboth{\nomname}{\nomname}% maybe with \MakeUppercase
\printnomenclature
\thispagestyle{plain}
\nomenclature{}{\pagestyle{plain}}


%% Ha a fordító nem fordítja bele a jelölésjegyzéket, akkor lehet használni az alábbi táblázatos módszert. A kettő közül csak az egyik maradjon a dolgozatban!


%\chapter*{Jel\"{o}l\'esjegyz\'ek}
% \chapter*{Jelölésjegyzék}
% \addcontentsline{toc}{chapter}{Jelölésjegyzék}
% \begin{longtable}{r p{13cm}}
%     $AI$:   & Artificial Intelligence (Mesterséges Intelligencia) \\
%     $GPU$:  & {Graphical Processing Unit (Grafikus Processzor / Grafikus Feldolgozó Egység)} \\
%     $API$:  & Application Programming Interface (Alkalmazásprogramozási Felület) \\
%     $CPU$:  & Central Processing Unit (Központi Feldolgozó Egység / Processzor) \\
%     $GUI$:  & Graphical User Interface (Grafikus Felhasználói Felület) \\
%     $HCI$:  & Human Computer Interaction (Ember-gép kapcsolat) \\
%     $CIS$:  & Cognitive Information System (Kognitív információs rendszer)
% \end{longtable}



%fordítás: terminálban:

%pdflatex document.tex
%makeindex document.nlo -s nomencl.ist -o document.nls
%pdflatex document.tex
