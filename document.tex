\documentclass[12pt,a4paper,oneside]{report}        % Single-side
%\documentclass[12pt,a4paper,twoside]{report}  		% Duplex

\usepackage[T1]{fontenc}
\usepackage[utf8]{inputenc}
\usepackage{amsmath}
\usepackage{amssymb}
\usepackage{enumerate}
\usepackage{graphicx}
\usepackage{lastpage}
\usepackage{anysize}
\newcommand\magyarOptions{chapterhead=unchanged}
\usepackage[magyar]{babel}
\usepackage{sectsty}
\usepackage{setspace}
\usepackage{hyperref}
\usepackage{fancyhdr}
\usepackage{titlesec}
\usepackage{pdfpages}
\usepackage[intoc]{nomencl}
\usepackage{lipsum}
\usepackage{float}
\usepackage{caption}
\usepackage{algorithm}
\usepackage{algpseudocode}
\usepackage[a-2b,mathxmp]{pdfx}
\usepackage{longtable}
\usepackage{url}
\usepackage{tikz}
\usepackage{listings}

\lstdefinelanguage{GDScript}{
  morekeywords={
    onready, tool, signal, extends, func, var, const, return,
    break, continue, if, elif, else, for, while, match, case, in, 
    and, or, not, pass, preload, is, as, self, true, false, null,
    export, setget, static, enum, class,
	func, if, is, true, false, emit_signal, MOUSE_BUTTON_LEFT, 
    InputEventMouseButton, new_label_text, String
  },
  sensitive=true,
  morecomment=[l]{\#},
  morestring=[b]",
}

\definecolor{codegreen}{rgb}{0,0.6,0}
\definecolor{codegray}{rgb}{0.5,0.5,0.5}
\definecolor{codepurple}{rgb}{0.58,0,0.82}
\definecolor{backcolour}{rgb}{0.95,0.95,0.92}

\lstdefinestyle{mystyle}{
	backgroundcolor=\color{backcolour},   
    commentstyle=\color{codegreen},
    keywordstyle=\color{magenta},
    numberstyle=\tiny\color{codegray},
    stringstyle=\color{codepurple},
    breakatwhitespace=false,         
    breaklines=true,                 
    captionpos=b,                    
    keepspaces=true,                 
    numbers=left,                    
    numbersep=5pt,                  
    showspaces=false,                
    showstringspaces=false,
    showtabs=false,                  
    tabsize=2
}
\lstset{
	style=mystyle,
    basicstyle=\ttfamily\footnotesize,
    basicstyle=\ttfamily,
    breaklines=true,
    numbers=left,
    numberstyle=\tiny,
    frame=single
}



\setlength{\parindent}{12pt}
\setlength{\parskip}{0pt}

\renewcommand{\baselinestretch}{1.5}
\titlespacing*{\section}
{0pt}{5.5ex plus 1ex minus .2ex}{4.3ex plus .2ex}

\marginsize{35.4mm}{25.4mm}{25.4mm}{25.4mm} % anysize package

\titleformat{\chapter}[display]{\fontsize{14}{15} \bfseries}{\thechapter. \chaptername}{15pt}{}
\sectionfont{\fontsize{12}{15}\bfseries}
\subsectionfont{\fontsize{12}{15}\bfseries}

\hypersetup{
	bookmarks=true,            % show bookmarks bar?
	unicode=false,             % non-Latin characters in Acrobat’s bookmarks
	pdftitle={},        % title
	pdfauthor={},    % author
	pdfsubject={}, % subject of the document
	pdfcreator={},   % creator of the document
	pdfproducer={Producer},    % producer of the document
	pdfkeywords={keywords},    % list of keywords
	pdfnewwindow=true,         % links in new window
	colorlinks=true,           % false: boxed links; true: colored links
	linkcolor=black,           % color of internal links
	citecolor=black,           % color of links to bibliography
	filecolor=black,           % color of file links
	urlcolor=black             % color of external links
}

\pagestyle{fancy}
\fancyhf{}
\fancyfoot[C]{\thepage}
\fancyhead[C]{\leftmark}

\fancypagestyle{plain}{%
	\fancyhf{}%
	\fancyfoot[C]{\thepage}%
	\fancyhead{}
	\renewcommand{\headrulewidth}{0pt}
}

\renewcommand{\chaptermark}[1]{%
	\markboth{#1}{}}
\renewcommand{\headrulewidth}{\iffloatpage{0pt}{0.4pt}}

\DeclareMathOperator*{\argmax}{arg\,max}
\DeclareMathOperator*{\argmin}{arg\,min}
%alairasok beszurasa nyilatkozatokra
\newcommand*{\SignatureAndDate}[1]{%
\par\noindent\makebox[3.5in]{}\hfill\makebox[2.0in]{\dotfill}%
\par\noindent\makebox[3.5in]{}\hfill\makebox[1.8in]{#1}
}%

\renewcommand{\nomname}{Jelölésjegyzék}

\makenomenclature
\makeindex

%%%%%%%%%%%%%%%%%%%%%%%%%%%%%%%%%%%%%%%%%%%%%%%%%%%%%%%%%%%%%%%%%%%%%%%%%%%%%%%%%%%%%%%%%%%%%%%%%%%%%
%%%%%% adatok: FONTOS, HOGY MINDEGYIK ADAT UTÁN LEGYEN SPACE! %%%%%%%%%%%%%%%%%%%%%%%%%%%%%%%%%%%%%%%
\def\myuni{Pannon Egyetem }
\def\mykar{Műszaki Informatikai Kar }
\def\mytanszek{\textbf{\textcolor{red}{<<tanszék>>}} } %többi tanszék?
\def\mycim{\textbf{\textcolor{red}{<<A szakdolgozat címe - a témakiírással egyezően>>}} }
\def\myszak{\textbf{\textcolor{red}{<<szak>>}} } %Mérnökinformatikus/Gazdaságinformatikus/Programtervező Informatikus/ Villamosmérnöki
\def\mynev{\textbf{\textcolor{red}{<<hallgató neve>>}} }
\def\myev{\textbf{\textcolor{red}{<<végzés éve>>}} }
\def\mytemavezeto{\textbf{\textcolor{red}{<<témavezető neve>>}} }
\def\mykonzulens{\textbf{\textcolor{red}{<<külső konzulens neve>>, <<intézménye>> }} }
\def\mydate{\textbf{\textcolor{red}{<<év. hónap nap>> }} }
\def\myvaros{\textbf{\textcolor{red}{<<hely>>}} }
\def\myvegzettseg{\textbf{\textcolor{red}{<<végzettség>>}} }
\def\mydolgozat{\textbf{\textcolor{red}{<<szak-/diplomadolgozat>>}} }

%%%%%%%%% kitöltendő - komment kitöRlése után ez kerül be a dolgozatba, és nem a piros fenti rész %%%%%%%%%%%%%%
%%%%%% FONTOS, HOGY MINDEGYIK ADAT UTÁN LEGYEN SPACE! %%%%%%%%%%%%%%%%%%%%%%%%%%%%%%%%%%%%%%%
%%%%%% A még nem biztos (pl pontos dátum az aláírásnál) részt érdemes még kikommentelve hagyni, így pirossal marad bent a végéig %%%%%%%%
\def\mycim{3D memóriajáték tervezése mesterséges intelligenciával }
\def\mynev{Csesznák Tamás Levente }
\def\myev{2024 }
\def\mytemavezeto{Szabó Patrícia }
%\def\mykonzulens{<<külső konzulens neve>>, <<intézménye>> }
\def\mydate{2024.05.01 }
\def\myvaros{Veszprém }


%%%%%%%%% tanszék - megfelelő elől kivenni a kommentet %%%%%%%%%%%%%%%%%%%%%%%
%\def\mytanszek{Alkalmazott Informatikai }
\def\mytanszek{Informatikai Rendszerek és Alkalmazásai }
%\def\mytanszek{Matematika }
%\def\mytanszek{Rendszer- és Számítástudományi }
%\def\mytanszek{Villamosmérnöki és Információs Rendszerek }


%%%%%%%%% szakok - megfelelő elől kivenni a kommentet %%%%%%%%%%%%%%%%%%%%%%%%%%%%%%%%%%%%
    %% Gazdaságinformatikus BSc
%\def\myszak{Gazdaságinformatikus BSc} \def\myvegzettseg{gazdaságinformatikus } \def\mydolgozat{SZAKDOLGOZAT}
    
    %% Mérnökinformatikus BSc
\def\myszak{Mérnökinformatikus BSc} \def\myvegzettseg{mérnökinformatikus } \def\mydolgozat{SZAKDOLGOZAT}
    
    %% Programtervező informatikus BSc
%\def\myszak{Programtervező informatikus BSc} \def\myvegzettseg{programtervező informatikus } \def\mydolgozat{SZAKDOLGOZAT}

    %% Villamosmérnöki BSc
%\def\myszak{Villamosmérnöki BSc} \def\myvegzettseg{villamosmérnöki } \def\mydolgozat{SZAKDOLGOZAT}

    %% Mérnökinformatikus MSc
%\def\myszak{Mérnökinformatikus MSc} \def\myvegzettseg{okleveles mérnökinformatikus } \def\mydolgozat{DIPLOMADOLGOZAT}

    %% Programtervező informatikus MSc
%\def\myszak{Programtervező informatikus MSc} \def\myvegzettseg{okleveles programtervező informatikus } \def\mydolgozat{DIPLOMADOLGOZAT}




%itt kerülnek becsatolásra a dolgozat egymást követő fejezetei
\begin{document}
	\pagenumbering{arabic}
	\onehalfspacing
	%a fedlappal kezdj
	%\begin{titlepage}
%    \begin{center}
%        \vspace*{\fill}
%        \Huge \textbf{SZAKDOLGOZAT}\\
%        \vspace{10cm}
%        \Large \textbf{Gyurácz Olivér}\\
%        \vspace{1.5cm}
%        \large 2021
%        \vspace*{\fill}
%    \end{center}
%\end{titlepage}
\begin{titlepage}
    \begin{center}
        \vspace*{\fill}
        \large \textbf{\myuni}\\
        \large \mykar\\
        \large \mytanszek Tanszék\\
        \large \myszak \\
        \vspace{2cm}
        \Huge \textbf{\mydolgozat}\\
        \vspace{2cm}
        %a szakdolgozat címe
        \Large \textbf{\mycim}\\
        \vspace{2cm}
        %a dolgozatot készítő hallgató neve
        \Large \textbf{\mynev}\\
        \vspace{2cm}
        %témavezetőjének a neve
        \large Témavezető: \mytemavezeto\\
        \vspace{1cm}
        %ha van, akkor kitöltendő, ha nincs, akkor törölje a következő sort
        %\large Külső/belső konzulens: \mykonzulens\\
        \vspace{1cm}
        % a dolgozat készítésének évszáma
        \large \myev
        \vspace*{\fill}
    \end{center}
\end{titlepage} 
	%a szkennelt témakiírás a következő, amit pdf-ben csatolunk
	\includepdf{Temakiiras.pdf}
	%saját nyilatkozatod nyomtatva, aláírva, pdf-be szkennelve csatolandó. A szerkeszthető részt tartalmazó  h_nyilatkozat.tex fájlt nem kell becsatolni, miután az aláírt nyilatkozat csatolásra kerül
	\begin{center}
\textbf{\large{Hallgatói nyilatkozat}}\\[32pt]
\end{center}

\thispagestyle{fancy}
\pagestyle{fancy}
Alulírott \mynev hallgató kijelentem, hogy a dolgozatot a \myuni \mytanszek Tanszékén készítettem a \myvegzettseg végzettség megszerzése érdekében.\par

Kijelentem, hogy a dolgozatban lévő érdemi rész saját munkám eredménye, az érdemi részen kívül csak a hivatkozott forrásokat (szakirodalom, eszközök stb.) használtam fel.\par

Tudomásul veszem, hogy a dolgozatban foglalt eredményeket a Pannon Egyetem, valamint a feladatot kiíró szervezeti egység saját céljaira szabadon felhasználhatja.\\


\vspace{2cm}
Dátum: \myvaros, \mydate\\
\vspace{2cm}

\SignatureAndDate{\mynev}

	%\includepdf{Hallgatoi_nyilatkozat.pdf}
	%témavezetőd nyilatkozata nyomtatva, aláírva, pdf-be szkennelve csatolandó. A szerkeszthető oldalt ki kell kapcsolni majd, ha a szkennelt témavezetői nyilatkozatot beszúrta. A beszúráshoz vegye le a % jelet a 94-es és 98-as sorok elől.
	\begin{center}
\textbf{\large{Témavezetői nyilatkozat}}\\[32pt]
\end{center}

\thispagestyle{fancy}
\pagestyle{fancy}
Alulírott \mytemavezeto témavezető kijelentem, hogy a dolgozatot \mynev a \myuni \mytanszek Tanszékén készítette a \myvegzettseg végzettség megszerzése érdekében.\\

Kijelentem, hogy a dolgozat védésre bocsátását engedélyezem.\\



\vspace{2cm}
Dátum: \myvaros, \mydate\\
\vspace{2cm}

\SignatureAndDate{\mytemavezeto}

	%\includepdf{Temavezetoi_nyilatkozat.pdf}
	%köszönetnyilvánítás csatolása
	%\textbf{\large{Köszönetnyilvánítás}}\\[32pt]

\thispagestyle{fancy}
\pagestyle{fancy}

Dolgozatom elkészültével szeretnék köszönetet mondani .....2-3 mondatban megfogalmazva.\\

Ezen kívül szeretnék köszönetet mondani e .. további köszönet akinek szeretné a Hallgató kifejezni.\\

Végül szeretném megköszönni a családomnak és a barátaimnak, amiért mindvégig önzetlenül támogattak célom elérésében.\\
\\

	%tartalmi összefoglalók csatolása, ami magyar és angol nyelven az Abstract.tex fájlban lesz megírva
	%\textbf{\large{Tartalmi összefoglaló}}\\[32pt]

\thispagestyle{fancy}
\pagestyle{fancy}



\vspace{8pt}

Tartalmi összefoglaló magyarul. Az összefoglalónak tartalmaznia kell (rövid, velős és összefüggő megfogalmazásban) a következőket:
\begin{itemize}
    \item téma megnevezése,
    \item megoldott feladat megfogalmazása,
    \item megoldási mód,
    \item elért eredmények,
    \item kulcsszavak (4-6 darab).
\end{itemize}
A tartalmi összefoglaló terjedelme nem lehet több egy A4-es oldalnál.\par
Az összefoglalót magyar és angol nyelven kell készíteni. Sorrendben a dolgozat nyelvével megegyező kerül előrébb. A cím Title stílusú, formázása: Times New Roman/ Computer Modern, nagybetű, 14 pt, félkövér, középre igazított; az összefoglaló szövege Normál stílusú, formázása: Times New Roman, 12 pt, sorkizárt, 1.5-ös sortávolság.

\vspace{8pt}



\textbf{Kulcsszavak: } Kulcsszó1, kulcsszó2, kulcsszó3, kulcsszó4, kulcsszó5, kulcszó6

\newpage



\textbf{\large{Abstract}}\\[32pt]

\thispagestyle{fancy}
\pagestyle{fancy}

\vspace{8pt}

Tartalmi összefoglaló angol nyelven, a tartalma és formázása megegyezik a magyar nyelvű tartalmi összefoglalóval.

\vspace{8pt}


\textbf{Keywords: } Keyword1, Keyword2, Keyword3, Keyword4, Keyword5, Keyword6 
	%Tartalomjegyzéket a Tex generálja, nem kell szerkeszteni
	\tableofcontents\vfill
	\addtocontents{toc}{\protect\thispagestyle{empty}}
	\pagestyle{empty}
	%rövidítések jegyzése a Jelolesjegyzek.tex fájlban szerkesztendő
	
%\nomenclature{$U$}{Feszültség}
%\nomenclature{$\boldsymbol{x}$}{Állapotvektor}
%\nomenclature{$\boldsymbol{A}$}{Incidencia mátrix}


\nomenclature{$AI$}{Artificial Intelligence - Mesterséges Intelligencia}
\nomenclature{$MI$}{Mesterséges Intelligencia}
% \nomenclature{$GPU$}{Graphical Processing Unit (Grafikus Processzor / Grafikus Feldolgozó Egység)}
\nomenclature{$API$}{Application Programming Interface (Alkalmazásprogramozási Felület)}
\nomenclature{$2D$}{Kettő dimenziós}
\nomenclature{$3D$}{Három dimenziós}
\nomenclature{$IDE$}{Integrált fejlesztői környezet}
\nomenclature{$ReLU$}{Rectified Linear Unit - Rektifikált lineáris egység}
\nomenclature{$mse$}{mean squared error - átlagos négyzetes hiba}
\nomenclature{$HTML$}{Hypertext Markup Language}
\nomenclature{$SUS$}{System Usability Scale}
\nomenclature{$HTML5$}{Hypertext Markup Language 5}
\nomenclature{$npm$}{Node Package Manager}
\nomenclature{$JSON$}{JavaScript Object Notation}
\nomenclature{$ID$}{Identification - azonosító}
\nomenclature{$NoSQL$}{Not Only SQL (Structured Query Language)}
\nomenclature{$IndexedDB$}{Indexed Data Base}
% \nomenclature{$CPU$}{Central Processing Unit (Központi Feldolgozó Egység / Processzor)}
% \nomenclature{$GUI$}{Graphical User Interface (Grafikus Felhasználói Felület)}
% \nomenclature{$HCI$}{Human Computer Interaction (Ember-gép kapcsolat)}
% \nomenclature{$CIS$}{Cognitive Information System (Kognitív információs rendszer)}



\cleardoublepage
\markboth{\nomname}{\nomname}% maybe with \MakeUppercase
\printnomenclature
\thispagestyle{plain}
\nomenclature{}{\pagestyle{plain}}



%fordítás: terminálban:

%pdflatex document.tex
%makeindex document.nlo -s nomencl.ist -o document.nls
%pdflatex document.tex

	%dolgozat fejezetei, amit a HALLGATÓ MEGÍR. A fejezetek tetszőleges néven létrehozhatók, fejezetnev.tex formátumban, a dokumentumba befordításuk az itt megadott sorrendben történik.
	\chapter{Bevezetés}
\usetikzlibrary{shapes,arrows}

\thispagestyle{fancy}
\pagestyle{fancy}
\section{Projekt célja}
A jelenlegi kor társadalmi és technológiai kihívásai közepette egyre fontosabbá válik az emberiség számára az olyan innovatív megoldások keresése, amelyek segíthetnek fejleszteni és támogatni az emberek mindennapi életét. Az Artificial Intelligence (AI), vagyis a Mesterséges Intelligencia, ebben az összefüggésben különösen figyelemre méltó tényezővé vált. Bár sokan aggódnak amiatt, hogy az AI alkalmazása az emberi társadalom hanyatlásához vezethet, én úgy vélem, hogy a megfelelő módon felhasználva az AI lehetőségei elősegíthetik a társadalmi fejlődést és előnyöket hozhatnak az emberi élet számos területén.

Szakdolgozatom központi célja az, hogy az AI alkalmazásával támogassam embertársaim rövidtávú memóriájának fejlesztését. Ehhez egy saját fejlesztésű virtuális valóság alapú, három dimenziós memóriajátékot tervezek létrehozni, amely segítségével interaktív és hatékony módon lehet fejleszteni a játékosok kognitív képességeit. 
\section{Projet bemutatása}
\begin{figure}[H]
    \centering
        \begin{tikzpicture}[>=latex',node distance=2cm,auto]
        % Define block styles
        \tikzstyle{block} = [rectangle, draw, text width=3cm, text centered, minimum height=1.5cm ]
        \tikzstyle{line} = [draw, -latex']

        % Nodes
        \node [block] (kutatomunka) {Kutatómunka};
        \node [block, below of=kutatomunka, node distance=3cm] (tervezes) {Játék megtervezése};
        \node [block, below of=tervezes, node distance=3cm] (jatek_megirasa) {Játék fejlesztése};
        \node [block, below of=jatek_megirasa, node distance=3cm] (adatgyujtes) {Adatgyűjtés};
        \node [block, below of=adatgyujtes, node distance=3cm] (AI) {AI betanítása};
        \node [block, below of=AI, node distance=3cm] (AI_game) {AI játszatása};
        \node [block, right of=AI_game, node distance=5cm] (AI_ember) {AI játszatása ember ellen};
        \node [block, right of=adatgyujtes, node distance=5cm] (VR) {VR támogatás};

        % \node [block, right of=section1, node distance=4cm] (subsection1) {Subsection 1};
        % \node [block, right of=section2, node distance=4cm] (subsection2) {Subsection 2};

        % Arrows
        \path [line] (kutatomunka) -- (tervezes);
        \path [line] (tervezes) -- (jatek_megirasa);
        \path [line] (jatek_megirasa) -- (VR);
        \path [line] (jatek_megirasa) -- (adatgyujtes);
        \path [line] (adatgyujtes) -- (AI);
        \path [line] (AI) -- (AI_game);
        \path [line] (AI) -- (AI_ember);

    \end{tikzpicture}
    \caption{Projekt folyamatábrája}
    \label{fig:folyamat_diagram}
\end{figure}
projektem több feladatból állt, melyet egy folyamatdiagram (\ref{fig:folyamat_diagram} ábra) szemléltet.

\subsection{Kutatómunka}
A kutatómunka során elsősorban azt vizsgáltam, hogy melyik tanító algoritmussal érhetem el a kívánt eredményt. Különböző irodalmakat tanulmányoztam, valamint áttekintettem mások munkáit a témában. A kutatómunka végeztével összegeztem a talált eredményeket.

\subsection{Játék megtervezése}
A kutatómunka után el kellett döntenem, hogy milyen játékot fejlesztek, amely elég bonyolult ahhoz, hogy kihívást jelentsen a játékosok számára, ugyanakkor elég egyszerű ahhoz, hogy az AI betanítása belátható időn belül megtörténjen. Ezen a ponton meg kellett azt is határoznom, hogy milyen technológiát alkalmazok, valamint hogy mely területekre összpontosítok a fejlesztés folyamán.

\subsection{Játék lefejlesztése}
A megfelelő tervezés után lefejlesztettem a választott fejlesztői környezetben a játékot. A fejlesztés során két fontos szempontot tartottam szem előtt: a játékot lehetővé kell tenni virtuális valóságban és asztali számítógépen egyaránt, valamint biztosítanom kell, hogy az AI képes legyen kezelni a játékot csupán a játék metainformációinak ismeretében.

\subsection{Adatgyűjtés és VR támogatás}
Miután elkészült a játék, több különböző korosztállyal játszattam azt annak érdekében, hogy elegendő adatom legyen az AI betanításához. Ebben az időszakban foglalkoztam a játék VR támogatásának fejlesztésével is.

\subsection{AI betanítása}
A gyűjtött adatokat felhasználva betanítottam az AI-t egy tanító algoritmus segítségével.

\subsection{AI játszatása}
A játékhoz létrehoztam egy interfészt, amely lehetővé tette az AI számára, hogy játszhasson vele. Miután ez sikeresen működött, lehetőséget teremtettem arra is, hogy az emberi játékos a gép ellen is játszhassa a játékot.
	%\chapter{Irodalomkutatás}

\thispagestyle{fancy}
\pagestyle{fancy}

\vspace{8pt}
\section{Új alfejezet}
\cite{tensor}
	\chapter{Felhasznált technológiák}

\thispagestyle{fancy}
\pagestyle{fancy}

Munkám során törekedtem arra, hogy a felhasznált technológiákat lehetőleg minimalizáljam. 
Figyelembe vettem továbbá azt is, hogy nyílt forráskódú, és multiplatform eszközöket válasszak. Ezen döntések lehetővé tették számomra a kellő flexibilitást, és elősegítették a munkámat. 
	\chapter{A játék működése}

\thispagestyle{fancy}
\pagestyle{fancy}
\section{Játék ismertetése}
A játék melyet lefeljleszettem, a közismert memória játék. A játékot lehet egyedül, vagy akár többen is játszani.

A játékban, egy asztalon 6, 8, 10 vagy 12 pár kártya található, képpel lefelé fordítva ahogyan az a \ref{img:asztal}. ábrán is látható.
A kártyák előlapján betűk találhatók. Egyjátékos esetben a játékos célja, hogy minél kevesebb kártyapár megfordításából megtalálja az összes memória párt. Többjátékos esetben, hogy ő szerezze a legtöbb pontot, vagyis több kártyapárt fordítson fel, mint az ellenfelei.

Ahhoz, hogy egy kártyát megfordítson, a játékosnak rá kell kattintania. Ekkor láthatóvá válik, mely betűhöz tartozik a memória elemhez (\ref{img:kartya_fliped}. ábra). 
A megfordított kártyához választani kell egy másikat. A játékosnak törekednie kell, hogy korábbi ismeretei alapján, hogy a következőre a választott kártya előlapján ugyanaz a betű szerepeljen, mint a már felfordított memória lapon, vagyis egy párt fordítson fel. 
Értelemszerűen ez az első felfordításkor nem lehetséges, hiszen nincs korábbi ismerete a játékról (\ref{img:non_pair}. ábra).

Ha a felfordított kártyák nem alkotnak párt, akkor a kártyák maguktól visszafordulnak pár másodperc elteltével. Ez után egyszemélyes játék esetén esetén végrehajtunk egy újabb fordítást. Többjátékos esetén a következő játékos végezheti el a körét. 

Ha párt alkotnak, akkor a kártyákat magunkhoz vehetjük. Ez mutatja a pontunkat, mely csak többjátékos esetben fog számítani.




\begin{figure}
    \includegraphics[width=\textwidth]{img/asztal_4x4.png}
    \caption{4x4-es memóriajáték kezdő állapota}
    \label{img:asztal}
\end{figure}
\begin{figure}
    \includegraphics[width=\textwidth]{img/asztal_4x4_card_flipped.png}
    \caption{4x4-es memóriajáték egy kártya ki van választva}
    \label{img:kartya_fliped}
\end{figure}
\begin{figure}
    \includegraphics[width=\textwidth]{img/asztal_4x4_non_pair.png}
    \caption{4x4-es memóriajáték. Mivel a betűk nem azonosak, ezér ez nem egy pár, visszafordítjuk a kártyákat.}
    \label{img:non_pair}
\end{figure}
	%\chapter{Összefoglalás}
\thispagestyle{fancy}
\pagestyle{fancy}


\section{Az egyes részfeladatok megvalósításának összefoglalása és értékelése}


	%Irodalomjegyzék generálása a dolgozatba a következő sorok hatására történik, szerkeszteni nem kell. A hivatkozott források gyűjteményét a mybib.bib nevű fájlban kell kialakítani
	\bibliography{mybib}
	\pagestyle{plain}
	\addcontentsline{toc}{chapter}{Irodalomjegyzék}
	%megjelenés sorrendjében építi fel az irodalomjegyzéket
	\bibliographystyle{ieeetr}
	% a következő sor a dolgozat digitális mellékletének a tartalmát csatolja a dolgozathoz, ezt a fájllistát a files.tex fájlban kell megírni
	%\chapter{Mellékletek}
\thispagestyle{fancy}
\pagestyle{fancy}

\thispagestyle{fancy}
\pagestyle{fancy}

\subsubsection{Beadott fájlok}
\noindent 3D-memoria-jatek-mestereseges-inteligenciaval mappa (a szoftver fájljai):\\
\begin{tabular}{l l}
    \quad icon.svg  & \quad Godot alapértelmezett ikonja a játékhoz \\ 
    \quad icon.svg.import  & \quad Godot metaadat \\ 
    \quad openxr\_action\_map.tres & \quad OpenXR metaadat \\ 
    \quad project.godot  & \quad Godot projekt fájl \\ 
\end{tabular}
\\\\
\noindent 3D-memoria-jatek-mestereseges-inteligenciaval/Animations mappa (a szoftver animációs fájlok):\\
\begin{tabular}{l l}
    \quad Idle.res  & \quad Kártya animáció \\ 
    \quad Megfordit.res  & \quad Kártya megfordít animáció \\ 
    \quad RESET.res  & \quad Kártya Reset animáció \\
\end{tabular}
\\\\
\noindent 3D-memoria-jatek-mestereseges-inteligenciaval/Scenes mappa (a szoftver jelenetei):\\
\begin{tabular}{l l} 
    \quad MainScrenes  & \quad Fő jelenet\\ 
    \quad MenuScene.tscn  & \quad Főmenü jelenet \\ 
    \quad XrOrigin.tscn  & \quad XR kezdőjelenet \\ 
    \quad button.tscn  & \quad Gomb jelenet \\ 
    \quad plainField.tscn  & \quad Üres tér jelenet \\ 
    \quad plain\_camera.tscn  & \quad Kamera jelenet \\ 
    \quad basics.tscn  & \quad Alapok jelenet \\ 
    \quad Card.gdshader  & \quad Kártya shader\\ 
    \quad Card.tscn  & \quad Kártya jelenet\\ 
    \quad Outline.gdshader  & \quad Kártya körvonal shader\\ 
\end{tabular}

\newpage

\noindent 3D-memoria-jatek-mestereseges-inteligenciaval/Scripts mappa (a szoftver GDScript szkriptjei):\\
\begin{tabular}{l l}
    \quad Card.gd  & \quad Kártya jelenet szkriptje \\ 
    \quad Deck.gd  & \quad Paklik jelenet szkriptje \\ 
    \quad InputField.gd  & \quad Bemeneti mező szkript \\ 
    \quad MenuScene.gd  & \quad Menü jelenet szkriptje \\ 
    \quad VrScript.gd  & \quad VR specifikus szkript \\ 
    \quad basics.gd  & \quad Alap jelent szrikptje \\ 
    \quad button.gd  & \quad Gomb jelent szriptje \\ 
    \quad constant.gd  & \quad Állandók definiálásra használt szkript \\ 
    \quad deck\_timer.gd  & \quad Pakli időzítő szkript \\ 
    \quad http\_client.gd  & \quad HTTP kérések kezelése szkript \\ 
    \quad player\_data.gd  & \quad Játékos adatainak tárolására használt adatszerkezet definiáló szript \\ 
    \quad scene\_manager.gd  & \quad Jelenet váltást kezelő szkript \\ 
\end{tabular}
\\\\
\noindent 3D-memoria-jatek-mestereseges-inteligenciaval/Shaders mappa (A játék grafikai megjelenítését szolgáló fájlok): \\
\begin{tabular}{l l}
    \quad MenuScene.tres  & \quad Menű kinézetét szolgáló shader könyvtár\\ 
    \quad cardTestLibary.tres  & \quad Egy teszt saját shader könyvtár tesztje \\ 
    \quad new\_standard\_material\_3d.tres  & \quad A Gombokhoz használt shader könyvtár \\     
\end{tabular}
\\\\
\noindent 3D-memoria-jatek-mestereseges-inteligenciaval/Templates mappa: \\
\begin{tabular}{l l}
    \quad custom\_build\_template.html  & \quad A HTML exporthoz használt saját html template\\ 
\end{tabular}
\\
\\
\noindent 3D-memoria-jatek-mestereseges-inteligenciaval/fileServer mappa: (Adatok gyüjtését szolgáló fájlszerver) \\
\begin{tabular}{l l}
    \quad Dockerfile  & \quad Docker konténer buildelését szolgáló fájl \\ 
    \quad docker-compose.yaml  & \quad Docker kontérek kezelését szolgáló compose fájl \\ 
    \quad package-lock.json  & \quad Node.JS package-lock fájl, npm használja\\ 
    \quad package.json  & \quad Node.JS package fájl, npm használja. \\ 
    \quad server.js  & \quad Fájlszerver forráskódja \\ 
\end{tabular}

\newpage

\noindent 3D-memoria-jatek-mestereseges-inteligenciaval/tensorflow: (Az MI modell használatához és tanításához szükséges fájlok.) \\
\begin{tabular}{l l}
    \quad json\_to\_sha256.py  & \quad JSON adatok SHA256 konvertálását szolgáló Python szkript \\ 
    \quad my\_model.keras  & \quad A betanított és  mentett MI modell \\ 
    \quad sha256\_to\_binary.py  & \quad SHA256 hash konvertálása bitekre Python szkript \\ 
    \quad tensor.py  & \quad TensorFlow tanító algoritmus, MI betanítása szkript \\ 
    \quad use\_model.py  & \quad Flask szerver, az MI modell használata Python szkript\\ 
\end{tabular}


\noindent p8mqg2\_tex mappa (tex fájlok és képek): \\
\begin{tabular}{l l}

\end{tabular}
\vspace{28pt}


	% a dolgozatban az ábrajegyzéket el kell helyezni, ezt a TeX automatikusan előállítja és a következő sor határása elhelyezi a dokumentumban
	\listoffigures
	%Ha vannak táblázatok is, akkor ezeket is listába gyűjtjük, és a következő utasítással az ábrajegyzéket követően a dolgozatba legeneráltatjuk. Ha nincs rá szükség, a százalék jellel a sor elején érvénytelenítjük a parancsot vagy töröljük a sort
	\listoftables
	
	
\end{document}
